\documentclass[11pt]{article}
\DeclareUnicodeCharacter{2212}{\ensuremath{-}}
\usepackage[russian]{babel}
\usepackage[utf8]{inputenc}
\usepackage[T2A]{fontenc}
\usepackage[russian,english]{babel}
\usepackage{amsmath}
\usepackage{amsfonts}
\usepackage{amssymb}
\usepackage[version=4]{mhchem}
\usepackage{stmaryrd}
\usepackage{soul}

\usepackage{geometry}
\geometry{
  a4paper,
  top=20mm, 
  right=20mm, 
  bottom=20mm, 
  left=20mm
}

\title{Линайная алгебра и геометрия. ИДЗ №4(вариант 21)}

\author{}
\date{}


\begin{document}
\maketitle

\section*{Выполнил студент Наседкин Дмитрий Сергеевич (группа 242)}

\subsection*{№1}
Нам даны 3 матрицы:

$$
A = \begin{pmatrix}
4 & 3 & 1 & -3 \\
-5 & -5 & -1 & 2 \\
-5 & -2 & -4 & 2 \\
9 & 4 & 1 & -7
\end{pmatrix}, \quad
B = \begin{pmatrix}
-4 & -3 & 1 & -1 \\
1 & 1 & -1 & 2 \\
1 & 4 & -4 & 2 \\
-1 & -4 & 1 & -5
\end{pmatrix} \quad \text{и} \quad
C = \begin{pmatrix}
-4 & -4 & -2 & 1 \\
1 & 3 & 3 & -2 \\
-1 & -8 & -7 & 3 \\
1 & 4 & 2 & -4
\end{pmatrix}
$$

Нужно проверить, что какие из них задают один и тот же линейный оператор, но в разных базисах, то есть проверить сопряжены ли матрицы. (то есть $A = D^{-1}BD$, где $D$ - невырожденная матрица). По утв. 131 это эквивалентно тому, что:

\begin{enumerate}
  \item Их характеристические многочлены равны между собой: $\chi_A(\lambda) = \chi_B(\lambda)$.
  \item Для любого корня $\lambda$ характеристического многочлена и любого числа $k$ не превосходящего кратности $\lambda$ выполнено $rk(A − \lambda E)^k = rk(B − \lambda E)^k$
\end{enumerate}

При этом заметим, что достаточно проверить все $k \leq \lfloor \frac{n_i}{2} \rfloor$, где $n_i$ - кратность корня. (Это следует из того, что по сути мы проверяем, что у них одинаковые жордановы клетки для одного собственного значения, но если что это было и на семе и на лекции).

Найдем их хар. многочлены:

Для матрицы $A$:

$$
\begin{gathered}
\det(A - x E) = \det
\begin{pmatrix}
4 - x & 3 & 1 & -3 \\
-5 & -5 - x & -1 & 2 \\
-5 & -2 & -4 - x & 2 \\
9 & 4 & 1 & -7 - x
\end{pmatrix}
= \det
\begin{pmatrix}
4 - x & 3 & 1 & -3 \\
-5 & -5 - x & -1 & 2 \\
0 & 3 + x & -3 - x & 0 \\
5 + x & 1 & 0 & -4 - x
\end{pmatrix} = \\[10pt]
= \det
\begin{pmatrix}
-1 - x & -2 - x & 0 & -1 \\
-5     & -5 - x & -1 & 2 \\
0      & 3 + x  & -3 - x & 0 \\
5 + x  & 1      & 0      & -4 - x
\end{pmatrix}
\end{gathered}
$$

Найдем определитель через формулу разложения по третьей строке:

$$
\begin{gathered}
det(A - x E) = -(3 + x)\det
\begin{pmatrix}
-1 - x & 0 & -1 \\
-5 & -1 & 2 \\
5 + x & 0 & -4 - x
\end{pmatrix} + (-3 - x)\det
\begin{pmatrix}
-1 - x & -2 - x & -1 \\
-5 & -5 - x & 2 \\
5 + x  & 1 & -4 - x
\end{pmatrix} = \\[6pt]
= (x + 3)(x^2 + 6x + 9) + (x + 3)(x^3 + 8x^2 + 21x + 18) = (x + 3)^3 + (x + 3)^3(x + 2) = (x + 3)^4.
\end{gathered}
$$

Я верю, что у матрицы $B, C$ такой же хар. многочлен(иначе это не интересная задача), да и видим, что у них следы совпадают, а значит сумма корней в хар. многочлене одинаковая, поэтому давайте лишь проверим наше предположение, для этого можно сделать следующее:

\begin{itemize}
  \item Проверим, что $\lambda = -3 \in \text{spec} \ B, \text{spec} \ C$
  \item Проверим, что $\chi_A(B) = 0$ и $\chi_A(C) = 0$.
\end{itemize}

Для $B$:

$$
\begin{gathered}
B + 3E =
\begin{pmatrix}
-4 + 3 & -3 & 1 & -1 \\
1 & 1 + 3 & -1 & 2 \\
1 & 4 & -4 + 3 & 2 \\
-1 & -4 & 1 & -5 + 3
\end{pmatrix} = 
\begin{pmatrix}
-1 & -3 & 1 & -1 \\
1 & 4 & -1 & 2 \\
1 & 4 & -1 & 2 \\
-1 & -4 & 1 & -2
\end{pmatrix} \Rightarrow B + 3E \text{ вырождена} \Rightarrow \\[6pt] \Rightarrow \lambda = -3 \in \text{spec} \ B
\end{gathered}
$$

$$
\chi_A(B) = (B + 3)^4 = 
\begin{pmatrix}
-1 & -3 & 1 & -1 \\
1 & 4 & -1 & 2 \\
1 & 4 & -1 & 2 \\
-1 & -4 & 1 & -2
\end{pmatrix}^4 =
\begin{pmatrix}
0 & -1 & 0 & -1 \\
0 & 1 & 0 & 1 \\
0 & 1 & 0 & 1 \\
0 & -1 & 0 & -1
\end{pmatrix}^2 = 
\begin{pmatrix}
0 & 0 & 0 & 0 \\
0 & 0 & 0 & 0 \\
0 & 0 & 0 & 0 \\
0 & 0 & 0 & 0
\end{pmatrix} \Rightarrow \chi_B(\lambda) = (\lambda + 3)^4
$$

Для $C$:

$$
\begin{gathered}
C + 3E =
\begin{pmatrix}
-4 + 3 & -4 & -2 & 1 \\
1 & 3 + 3 & 3 & -2 \\
-1 & -8 & -7 + 3 & 3 \\
1 & 4 & 2 & -4 + 3
\end{pmatrix} = 
\begin{pmatrix}
-1 & -4 & -2 & 1 \\
1 & 6 & 3 & -2 \\
-1 & -8 & -4 & 3 \\
1 & 4 & 2 & -1
\end{pmatrix} 
\Rightarrow C + 3E \text{ вырождена} \Rightarrow \\[6pt] \Rightarrow \lambda = -3 \in \text{spec} \ C
\end{gathered}
$$

$$
\chi_A(C) = (C + 3)^4 = 
\begin{pmatrix}
-1 & -4 & -2 & 1 \\
1 & 6 & 3 & -2 \\
-1 & -8 & -4 & 3 \\
1 & 4 & 2 & -1
\end{pmatrix}^4 =
\begin{pmatrix}
0 & 0 & 0 & 0 \\
0 & 0 & 0 & 0 \\
0 & 0 & 0 & 0 \\
0 & 0 & 0 & 0
\end{pmatrix}^2 = 
\begin{pmatrix}
0 & 0 & 0 & 0 \\
0 & 0 & 0 & 0 \\
0 & 0 & 0 & 0 \\
0 & 0 & 0 & 0
\end{pmatrix} \Rightarrow \chi_C(\lambda) = (\lambda + 3)^4
$$

То есть первый пункт из утверждения для всех матриц верный, проверим теперь второй(то есть будем сверять ранги):

Вообще, уже видно, что $C$ и $B$ точно не задают один линейный оператор, так как $rk(C + 3E)^2 = 0 \neq rk(B + 3E)^2$.

$$
rk(A + 3E) = rk
\begin{pmatrix}
7 & 3 & 1 & -3 \\
-5 & -2 & -1 & 2 \\
-5 & -2 & -1 & 2 \\
9 & 4 & 1 & -4
\end{pmatrix} = 2
$$

$$
rk(A + 3E)^2 = rk
\begin{pmatrix}
7 & 3 & 1 & -3 \\
-5 & -2 & -1 & 2 \\
-5 & -2 & -1 & 2 \\
9 & 4 & 1 & -4
\end{pmatrix}^2 = rk
\begin{pmatrix}
2 & 1 & 0 & -1 \\
-2 & -1 & 0 & 1 \\
-2 & -1 & 0 & 1 \\
2 & 1 & 0 & -1
\end{pmatrix} = 1
$$

Отсюда видно, что $A$ и $C$ точно не задают один линейный оператор, а вот $rk(B + 3E) = 2, rk(B + 3E)^2 = 1$, а значит по утв. $A, B$ действительно задают один лин. оператор.

\subsection*{№2}
Видим блок нулей, а значит будет удобнее работать с нашей матрицей $6 \times 6$ как с блоками(назовем исходную матрицу $M$).

$$
M=\begin{pmatrix}
A & 0\\
B & C
\end{pmatrix},
$$

где

$$
A =
\begin{pmatrix}
4 & -3 & 2 \\
0 & 2 & 1 \\
0 & -2 & 5
\end{pmatrix}, \quad
B =
\begin{pmatrix}
-2 & 2 & 1 \\
-4 & 8 & 1 \\
2 & -4 & 2
\end{pmatrix}, \quad
C =
\begin{pmatrix}
2 & 1 & 0 \\
-4 & 6 & 0 \\
-1 & 1 & 3
\end{pmatrix}.
$$

Для начала найдем $\chi_M(\lambda) = det(M - \lambda E) = det \begin{pmatrix}A - \lambda E & 0 \\ B & C - \lambda E \end{pmatrix} = det(A - \lambda E) det(C - \lambda E)$. Последнее равенство верно как раз из-за угла нулей (док-во легко делается или обычным рассуждением из явной формулы, либо можно юзануть утв. 25 с лекций).

Раскроем определитель по формуле разложения по столбцу(для $A$ по первому, для $C$ по третьему):

$$
\begin{gathered}
det(A - \lambda E) = det
\begin{pmatrix}
4 - \lambda & -3 & 2 \\
0 & 2 - \lambda & 1 \\
0 & -2 & 5 - \lambda
\end{pmatrix} = (4 - \lambda)((2 - \lambda)(5 - \lambda) + 2) = (4 - \lambda)(10 - 7 \lambda + \lambda^2 + 2) = \\[6pt] = (4 - \lambda)(\lambda^2 - 7 \lambda + 12) = -(\lambda - 4)^2(\lambda - 3)
\end{gathered}
$$

$$
det(C - \lambda E) = det
\begin{pmatrix}
2 - \lambda & 1 & 0 \\
-4 & 6 - \lambda & 0 \\
-1 & 1 & 3 - \lambda
\end{pmatrix} = (3 - \lambda)((2 - \lambda)(6 - \lambda) + 4) = (3 - \lambda)(\lambda^2 - 8\lambda + 16) = (3 - \lambda)(\lambda - 4)^2
$$

Таким образом, $\chi_M(\lambda) = (\lambda - 3)^2(\lambda - 4)^4$.

\noindent\rule{\textwidth}{0.5pt}

То есть мы знаем, что в ЖНФ суммарный размер всех клеток с $\lambda = 3$ равен 2, а суммарный размер клеток с $\mu = 4$ равен 4. Давайте вычислим число клеток с $\lambda$ и $\mu$, для этого нужно вычислить $\dim \ker(M - \lambda E)$ и $\dim \ker(M - \mu E)$ соответственно.

$$
\begin{gathered}
M - 3E =
\begin{pmatrix}
1 & -3 & 2 & 0 & 0 & 0 \\
0 & -1 & 1 & 0 & 0 & 0 \\
0 & -2 & 2 & 0 & 0 & 0 \\
-2 & 2 & 1 & -1 & 1 & 0 \\
-4 & 8 & 1 & -4 & 3 & 0 \\
2 & -4 & 2 & -1 & 1 & 0
\end{pmatrix} \sim
\begin{pmatrix}
1 & -3 & 2 & 0 & 0 & 0 \\
0 & -1 & 1 & 0 & 0 & 0 \\
0 & 0 & 1 & -1 & 1 & 0 \\
0 & 0 & 0 & 1 & -2 & 0 \\
0 & 0 & 0 & 0 & -1 & 0 \\
0 & 0 & 0 & 0 & 0 & 0
\end{pmatrix} \Rightarrow \\[6pt] \Rightarrow \dim \ker(M - \lambda E) = 1 \Rightarrow \text{В ЖНФ одна клетка со значением 3 размера 2}.
\end{gathered}
$$

$$
\begin{gathered}
M - 4E =
\begin{pmatrix}
0 & -3 & 2 & 0 & 0 & 0 \\
0 & -2 & 1 & 0 & 0 & 0 \\
0 & -2 & 1 & 0 & 0 & 0 \\
-2 & 2 & 1 & -2 & 1 & 0 \\
-4 & 8 & 1 & -4 & 2 & 0 \\
2 & -4 & 2 & -1 & 1 & -1
\end{pmatrix} \sim
\begin{pmatrix}
-2 & 2 & 1 & -2 & 1 & 0 \\
0 & 4 & -1 & 0 & 0 & 0 \\
0 & -2 & 3 & -3 & 2 & -1 \\
0 & -3 & 2 & 0 & 0 & 0 \\
0 & -2 & 1 & 0 & 0 & 0
\end{pmatrix} \sim
\begin{pmatrix}
-2 & 2 & 1 & -2 & 1 & 0 \\
0 & 1 & 0 & 0 & 0 & 0 \\
0 & 0 & 1 & 0 & 0 & 0 \\
0 & 0 & 1 & 0 & 0 & 0 \\
0 & 0 & 2 & -3 & 2 & -1
\end{pmatrix} \Rightarrow \\[6pt] \Rightarrow \dim \ker(M - \mu E) = 2 \Rightarrow \text{В ЖНФ клетки со значением 4 это либо 1-3, либо 2-2}.
\end{gathered}
$$

Посчитаем кол-во клеток размера 1 со значением 4 по формуле: (утв. 130)

$$
\begin{gathered}
2 \dim \ker(M - \mu E)^1 - \dim \ker(M - \mu E)^2 - \dim \ker(M - \mu E)^0 = 2 * 2 - \dim \ker(M - \mu E)^2 = \\ = 4 - \dim \ker(M - \mu E)^2 = 4 - 4 = 0 \Rightarrow \text{В ЖНФ клетки со значением 4 это 2 клетки размера 2}.
\end{gathered}
$$

Расчеты $\dim \ker(M - \mu E)^2$:

$$
\begin{gathered}
(M - \mu E)^2 =
\begin{pmatrix}
A - \mu E & 0 \\
B & C - \mu E
\end{pmatrix}^2 =
\begin{pmatrix}
(A - \mu E)^2 & 0 \\
B(A - \mu E) + (C - \mu E)B & (C - \mu E)^2
\end{pmatrix} = \\[6pt] =
\begin{pmatrix}
0 & 2 & -1 & 0 & 0 & 0 \\
0 & 2 & -1 & 0 & 0 & 0 \\
0 & 2 & -1 & 0 & 0 & 0 \\
0 & 4 & -2 & 0 & 0 & 0 \\
0 & 2 & -1 & 0 & 0 & 0 \\
-4 & 8 & 0 & -1 & 0 & 1
\end{pmatrix} \Rightarrow \dim \ker(M - \mu E)^2 = 4
\end{gathered}
$$

Таким образом ЖНФ:

$$
J =
\begin{pmatrix}
3 & 1 & 0 & 0 & 0 & 0 \\
0 & 3 & 0 & 0 & 0 & 0 \\
0 & 0 & 4 & 1 & 0 & 0 \\
0 & 0 & 0 & 4 & 0 & 0 \\
0 & 0 & 0 & 0 & 4 & 1 \\
0 & 0 & 0 & 0 & 0 & 4
\end{pmatrix}
$$

\noindent\rule{\textwidth}{0.5pt}

Теперь найдем жорданов базис:

Для $\lambda$ чтобы найти $f_1$ давайте найдем какое-нибудь решение системы $(M - \lambda E)f_1 = 0$:

$$
\begin{gathered}
M - \lambda E =
\begin{pmatrix}
1 & -3 & 2 & 0 & 0 & 0 \\
0 & -1 & 1 & 0 & 0 & 0 \\
0 & -2 & 2 & 0 & 0 & 0 \\
-2 & 2 & 1 & -1 & 1 & 0 \\
-4 & 8 & 1 & -4 & 3 & 0 \\
2 & -4 & 2 & -1 & 1 & 0
\end{pmatrix} \sim
\begin{pmatrix}
1 & -3 & 2 & 0 & 0 & 0 \\
0 & -1 & 1 & 0 & 0 & 0 \\
0 & 0 & 1 & -1 & 1 & 0 \\
0 & 0 & 0 & 1 & -2 & 0 \\
0 & 0 & 0 & 0 & -1 & 0 \\
0 & 0 & 0 & 0 & 0 & 0
\end{pmatrix} \Rightarrow \text{ ФСР системы } = (0, 0, 0, 0, 0, 1)^T \Rightarrow \\[6pt] \Rightarrow f_1 = (0, 0, 0, 0, 0, 1)^T
\end{gathered}
$$

Тогда чтобы найти $f_2$ нужно решить систему $(M - \lambda E)f_2 = f_1$:

$$
\begin{gathered}
\begin{pmatrix}
\begin{array}{c c c c c c|c}
1 & -3 & 2 & 0 & 0 & 0 & 0 \\
0 & -1 & 1 & 0 & 0 & 0 & 0 \\
0 & -2 & 2 & 0 & 0 & 0 & 0 \\
-2 & 2 & 1 & -1 & 1 & 0 & 0 \\
-4 & 8 & 1 & -4 & 3 & 0 & 0 \\
2 & -4 & 2 & -1 & 1 & 0 & 1
\end{array}
\end{pmatrix} \sim
\begin{pmatrix}
\begin{array}{c c c c c c|c}
1 & -3 & 2 & 0 & 0 & 0 & 0 \\
0 & -1 & 1 & 0 & 0 & 0 & 0 \\
0 & 0 & 1 & -1 & 1 & 0 & 0 \\
0 & 0 & 0 & 1 & -2 & 0 & 0 \\
0 & 0 & 0 & 0 & -1 & 0 & 1 \\
\end{array}
\end{pmatrix} \sim
\begin{pmatrix}
\begin{array}{c c c c c c|c}
1 & 0 & 0 & 0 & 0 & 0 & -1 \\
0 & -1 & 0 & 0 & 0 & 0 & 1 \\
0 & 0 & 1 & 0 & 0 & 0 & -1 \\
0 & 0 & 0 & 1 & 0 & 0 & -2 \\
0 & 0 & 0 & 0 & -1 & 0 & 1 \\
\end{array}
\end{pmatrix} \Rightarrow \\[6pt] \Rightarrow f_2 = (-1, -1, -1, -2, -1, 0)^T
\end{gathered}
$$

Теперь будем действовать аналогично для $\mu$ и таким образом найдем $f_3, f_4, f_5, f_6$:

$$
\begin{gathered}
M - \mu E =
\begin{pmatrix}
0 & -3 & 2 & 0 & 0 & 0 \\
0 & -2 & 1 & 0 & 0 & 0 \\
0 & -2 & 1 & 0 & 0 & 0 \\
-2 & 2 & 1 & -2 & 1 & 0 \\
-4 & 8 & 1 & -4 & 2 & 0 \\
2 & -4 & 2 & -1 & 1 & -1
\end{pmatrix} \sim
\begin{pmatrix}
-2 & 2 & 1 & -2 & 1 & 0 \\
0 & 1 & 0 & 0 & 0 & 0 \\
0 & 0 & 1 & 0 & 0 & 0 \\
0 & 0 & 2 & -3 & 2 & -1
\end{pmatrix} \sim
\begin{pmatrix}
-2 & 2 & 1 & -2 & 1 & 0 \\
0 & 1 & 0 & 0 & 0 & 0 \\
0 & 0 & 1 & 0 & 0 & 0 \\
0 & 0 & 0 & -3 & 2 & -1
\end{pmatrix} \sim \\[6pt] \sim
\begin{pmatrix}
6 & 0 & 0 & 0 & 1 & -2 \\
0 & 1 & 0 & 0 & 0 & 0 \\
0 & 0 & 1 & 0 & 0 & 0 \\
0 & 0 & 0 & 3 & -2 & 1
\end{pmatrix}
\end{gathered}
$$

Тогда ФСР:

$$
\begin{array}{}
f_3 = (*, *, *, *, 6, 0)^T = (-1, 0, 0, 4, 6, 0)^T \\[6pt]
f_5 = (*, *, *, *, 0, 6)^T = (1, 0, 0, -1, 0, 3)^T
\end{array}
$$

Решим $(M - \mu E)f_4 = f_3$ и $(M - \mu E)f_6 = f_5$:

$$
\begin{gathered}
\begin{pmatrix}
\begin{array}{c c c c c c|c c}
0 & -3 & 2 & 0 & 0 & 0 & -1 & 1\\
0 & -2 & 1 & 0 & 0 & 0 & 0 & 0\\
0 & -2 & 1 & 0 & 0 & 0 & 0 & 0\\
-2 & 2 & 1 & -2 & 1 & 0 & 4 & -1\\
-4 & 8 & 1 & -4 & 2 & 0 & 6 & 0\\
2 & -4 & 2 & -1 & 1 & -1 & 0 & 3
\end{array}
\end{pmatrix} \sim
\begin{pmatrix}
\begin{array}{c c c c c c|c c}
-2 & 2 & 1 & -2 & 1 & 0 & 4 & -1 \\
0 & -2 & 1 & 0 & 0 & 0 & 0 & 0 \\
0 & -3 & 2 & 0 & 0 & 0 & -1 & 1 \\
0 & 4 & -1 & 0 & 0 & 0 & -2 & 2 \\
0 & -2 & 3 & -3 & 2 & -1 & 4 & 2
\end{array}
\end{pmatrix} \sim \\[6pt] \sim
\begin{pmatrix}
\begin{array}{c c c c c c|c c}
-2 & 2 & 1 & -2 & 1 & 0 & 4 & -1 \\
0 & -2 & 1 & 0 & 0 & 0 & 0 & 0 \\
0 & 0 & 1 & 0 & 0 & 0 & -2 & 2 \\
0 & 0 & 1 & 0 & 0 & 0 & -2 & 2 \\
0 & 0 & 2 & -3 & 2 & -1 & 4 & 2
\end{array}
\end{pmatrix} \sim
\begin{pmatrix}
\begin{array}{c c c c c c|c c}
-2 & 2 & 1 & -2 & 1 & 0 & 4 & -1 \\
0 & -2 & 1 & 0 & 0 & 0 & 0 & 0 \\
0 & 0 & 1 & 0 & 0 & 0 & -2 & 2 \\
0 & 0 & 0 & -3 & 2 & -1 & 8 & -2 \\
\end{array}
\end{pmatrix} \sim \\[6pt] \sim
\begin{pmatrix}
\begin{array}{c c c c c c|c c}
1 & 0 & 0 & 0 & \tfrac{1}{6} & -\tfrac{1}{3} & -\tfrac{4}{3} & \tfrac{11}{6}\\[6pt]
0 & 1 & 0 & 0 & 0 & 0 & -1 & 1 \\[6pt]
0 & 0 & 1 & 0 & 0 & 0 & -2 & 2 \\[6pt]
0 & 0 & 0 & 1 & -\tfrac{2}{3} & \tfrac{1}{3}  & -\tfrac{8}{3} & \tfrac{2}{3}
\end{array}
\end{pmatrix} \Rightarrow \begin{gathered}
f_4 = (\frac{-4}{3}, -1, -2, \frac{-8}{3}, 0, 0)^T \\[6pt]
f_6 = (\frac{11}{6}, 1, 2, \frac{2}{3}, 0, 0)^T
\end{gathered}
\end{gathered}
$$

Таким образом матрица перехода имеет вид:

$$
S =
\begin{pmatrix}
0 & -1 & -1 & -\frac{4}{3} & 1 & \frac{11}{6}\\[6pt]
0 & -1 & 0 & -1 & 0 & 1\\[6pt]
0 & -1 & 0 & -2 & 0 & 2\\[6pt]
0 & -2 & 4 & -\frac{8}{3} & -1 & \frac{2}{3}\\[6pt]
0 & -1 & 6 & 0 & 0 & 0\\[6pt]
1 & 0 & 0 & 0 & 3 & 0
\end{pmatrix}
$$

\subsection*{№3}
\textbf{Переход в стандартный базис}:

Найдем матрицы перехода:

$$
G = (g_1 \;|\; g_2 \;|\; g_3)
=\begin{pmatrix}
-1 & -2 & -1\\
-1 & -3 & -2\\
-1 & -4 & -4
\end{pmatrix},\quad
H = (f_1 \;|\; f_2 \;|\; f_3)
=\begin{pmatrix}
1 & -1 & 1\\
-2 & 3  & -4\\
-1 & 2  & -4
\end{pmatrix}.
$$

Найдем обратные к ним матрицы:

$$
G^{-1} = \frac{\hat{G}}{\det(G)} = \frac{1}{-4+4-1}
\begin{pmatrix}
4 & -4 & 1\\
-2 & 3 & -1\\
1 & -2 & 1
\end{pmatrix} = 
\begin{pmatrix}
-4 & 4 & -1 \\
2 & -3 & 1 \\
-1 & 2 & -1
\end{pmatrix}
$$

$$
H^{-1} = \frac{\hat{H}}{\det(H)} = -
\begin{pmatrix}
-4 & -2 & 1 \\
-4 & -3 & 2 \\
-1 & -1 & 1
\end{pmatrix} = 
\begin{pmatrix}
4 & 2 & -1 \\
4 & 3 & -2 \\
1 & 1 & -1
\end{pmatrix}
$$

Найдем матрицы операторов в стандартном базисе:

$$
A_e \;=\; G\,A\,G^{-1},
\qquad
B_e \;=\; H\,B\,H^{-1},
$$

$$
A_e
=\begin{pmatrix}
-1 & -2 & -1 \\
-1 & -3 & -2 \\
-1 & -4 & -4
\end{pmatrix}
\begin{pmatrix}
4 & 7 & 0 \\
-2 & -1 & 5 \\
2 & 2 & -3
\end{pmatrix}
\begin{pmatrix}
-4 & 4 & -1 \\
2 & -3 & 1 \\
-1 & 2 & -1
\end{pmatrix} =
\begin{pmatrix}
-2 & -7 & -7 \\
-2 & -8 & -9 \\
-4 & -11 & -8
\end{pmatrix}
\begin{pmatrix}
-4 & 4 & -1 \\
2 & -3 & 1 \\
-1 & 2 & -1
\end{pmatrix} =
\begin{pmatrix}
1 & -1 & 2 \\
1 & -2 & 3 \\
2 & 1 & 1
\end{pmatrix}
$$

$$
B_e =
\begin{pmatrix}
1 & -1 & 1 \\
-2 & 3 & -4 \\
-1 & 2 & -4
\end{pmatrix}
\begin{pmatrix}
3 & -1 & -6 \\
3 & -5 & 6 \\
1 & -3 & 6
\end{pmatrix}
\begin{pmatrix}
4 & 2 & -1 \\
4 & 3 & -2 \\
1 & 1 & -1
\end{pmatrix} =
\begin{pmatrix}
1 & 1 & -6 \\
-1 & -1 & 6 \\
-1 & 3 & -6
\end{pmatrix}
\begin{pmatrix}
4 & 2 & -1 \\
4 & 3 & -2 \\
1 & 1 & -1
\end{pmatrix} =
\begin{pmatrix}
2 & -1 & 3 \\
-2 & 1 & -3 \\
2 & 1 & 1
\end{pmatrix}
$$

\noindent\rule{\textwidth}{0.5pt}
\textbf{Матрица композиции $\varphi\circ\psi$ в стандартном базисе}:

Если матрицы операторов для одного базиса, то композиция будет соответствовать:

$$
[\varphi\circ\psi]_e
= A_eB_e = \begin{pmatrix}
1 & -1 & 2 \\
1 & -2 & 3 \\
2 & 1 & 1
\end{pmatrix}
\begin{pmatrix}
2 & -1 & 3 \\
-2 & 1 & -3 \\
2 & 1 & 1
\end{pmatrix} =
\begin{pmatrix}
8 & 0 & 8 \\
12 & 0 & 12 \\
4 & 0 & 4
\end{pmatrix}
$$

\noindent\rule{\textwidth}{0.5pt}
Далее алгоритм следующий:

\begin{itemize}
  \item Найдем для каждого из операторов ее образ и ядро
  \item Выполним пункт б
  \item Выполним пункт в
\end{itemize}

\textbf{Замечание}:

Делать все это именно для операторов в одном базисе(В нашем случае в стандартном).

\noindent\rule{\textwidth}{0.5pt}

\textbf{$\ker\varphi$ и $\ker\psi$}:

Нужно найти ФСР системы $A_ex=0$ и $B_ex=0$, ФСР и будет базисом ядра:

$$
A_e = 
\begin{pmatrix}
1 & -1 & 2 \\
1 & -2 & 3 \\
2 & 1 & 1
\end{pmatrix} \sim
\begin{pmatrix}
1 & -1 & 2 \\
0 & -1 & 1 \\
\end{pmatrix} \sim
\begin{pmatrix}
1 & 0 & 1 \\
0 & 1 & -1 \\
\end{pmatrix}
$$

Тогда $\ker \varphi = \langle (-1, \; 1, \; 1) \rangle$:

$$
B_e =
\begin{pmatrix}
2 & -1 & 3 \\
-2 & 1 & -3 \\
2 & 1 & 1
\end{pmatrix} \sim
\begin{pmatrix}
2 & -1 & 3 \\
0 & 2 & -2
\end{pmatrix} \sim
\begin{pmatrix}
1 & 0 & 1 \\
0 & 1 & -1
\end{pmatrix}
$$

Тогда $\ker \psi = \langle (-1, \; 1, \; 1) \rangle$:

\noindent\rule{\textwidth}{0.5pt}

\textbf{$Im \ \varphi$ и $Im \ \psi$}:

Столбцы $A_e$ порождают $Im \ \varphi$, а $B_e$ — $Im \ \psi$.\\
Тогда базис $Im \ \varphi$ равен $(1, \; 1, \; 2), (-1, \; -2, \; 1)$ (первый и второй столбец $A_e$), а базис $Im \ \psi$ равен $(2, \; -2, \; 2), (-1, \; 1, \; 1)$ (тоже первый и второй столбец, только теперь $B_e$).

\noindent\rule{\textwidth}{0.5pt}

\textbf{Базис пространства $\ker\varphi + \ker\psi$}

Так как $\ker \varphi = \ker \psi = \langle (-1, \; 1, \; 1) \rangle \Rightarrow \ker \varphi + \ker \psi = \langle (-1, \; 1, \; 1) \rangle$.

\noindent\rule{\textwidth}{0.5pt}

\textbf{Базис пространства $Im \ \varphi\cap Im \ \psi$}

Воспользуемся алгоритмом из гита Димы(№ 13):

$$
v_1 = (1, \; 1, \; 2)^T,\; v_2 = (-1, \; -2, \; 1)^T,
\qquad
u_1 = (2, \; -2, \; 2)^T,\; u_2 = (-1, \; 1, \; 1)^T
$$

Собираем систему

$$
\bigl(v_1\;v_2\;|\;u_1\;u_2\bigr)
\begin{pmatrix}
\alpha_1 \\
\alpha_2 \\
\beta_1 \\
\beta_2
\end{pmatrix}
= 0
$$

и находим ФСР:

$$
\begin{pmatrix}
1 & -1 & 2 & -1\\
1 & -2 & -2 & 1\\
2 & 1 & 2 & 1
\end{pmatrix} \sim
\begin{pmatrix}
1 & -1 & 2 & -1 \\
0 & -1 & -4 & 2 \\
0 & 0 & -14 & 9
\end{pmatrix}
$$

$$
x = (-12, \; -8, \; 9, \; 14)^T
$$

$(\beta_1,\beta_2)=(9 ,\; 14)$.\\
Тогда

$$
R' \;=\; [\,u_1\;u_2\,]\,
\begin{pmatrix}
9\\
14
\end{pmatrix}
= 9u_1 + 14u_2
= 9(2, \; -2, \; 2)^T + 14(-1, \; 1, \; 1)^T
= (4, \, -4, \, 32)^T.
$$

Получили ровно \textbf{один} вектор, значит $\dim(Im \ \varphi \cap Im \ \psi)=1$, а значит он базисный. То есть:

$$
Im \ \varphi \cap Im \ \psi= \langle (1, \, -1, \, 8)^T \rangle.
$$

\subsection*{№4}
Нам дана матрица:

$$
A_t =
\begin{pmatrix}
-2 & 0 & 1 & t - 1 & 0 \\
0 & -2 & 0 & 0 & 0 \\
0 & t & -2 & 0 & 1 + t \\
0 & 0 & 0 & 3 & 0 \\
0 & 0 & 0 & 1 & 3
\end{pmatrix}
$$

Не зря задачи все таки на ЖНФ, поэтому давайте приведем $A_t$ к ЖНФ, тогда получим, что $A_t = PJP^{-1}$, где $P$ - матрица перехода от стандартного базиса к тому, в котором $J$ - ЖНФ.

Давайте \st{зачем-то} найдем матрицы все матрицы $D$, которые коммутируют с J(потому что их легко найти) и диагонализуемые, тогда какая связь вообще с тем, что $[A_t, X] = 0$ (коммутатор), и $X$ - диагональна?

Так как мы нашли $D$, которые записаны в том же базисе что и $J$, то теперь нужно перейти к стандартному базису, чтобы вернуться к $A_t$, то есть $X = PDP^{-1}$. Тогда если мы докажем, что $[A_t, X] = 0$ и $X$ - диагонализуема(и в обратную сторону), то получим, что $\dim L_t = \dim K_t$, где $K_t$ - подпространство матриц порожденное всеми нашими подходящими $D$.

\subsubsection*{1) Диагонализуемость $X \iff$ диагонализуемость $D$}
Матрицы сопряжены, значит у них одна структура ЖНФ, а значит из диагонализуемости одной следует диагонализуемость другой. 

\subsubsection*{2) $[A_t, X] = 0 \iff [J, D] = 0$}

$$
\begin{align}
\Longleftarrow{:}\quad 
  &XA_t = (PDP^{-1})A_t 
    = PD\bigl(P^{-1}A_tP\bigr)P^{-1}
    = PDJP^{-1}
    = PJDP^{-1}
    = (PJP^{-1})(PDP^{-1})
    = A_tX,\\
\Longrightarrow{:}\quad 
  &DJ = (P^{-1}XP)J
    = P^{-1}X\bigl(PJP^{-1}\bigr)P
    = P^{-1}XA_tP
    = P^{-1}A_tXP
    = (P^{-1}A_tP)(P^{-1}XP)
    = JD.
\end{align}
$$

\noindent\rule{\textwidth}{0.5pt}
Найдем ЖНФ $A_t$:

$\chi_{A_t}(\lambda) = (\lambda + 2)^3(\lambda - 3)^2$ (ищется совсем тривиально).

То есть у нас суммарных размер клеток с $\lambda = -2$ равен 3, а с $\lambda = 3$ равен 2. $\dim \ker(A_t - 3E) = 1 \Rightarrow$ для $\lambda = 3$ имеем один жорданов блок размера 2. С $\dim \ker(A_t + 2E)$ чуть сложней:

$$
A_t + 2E =
\begin{pmatrix}
0 & 0 & 1 & t-1 & 0 \\
0 & 0 & 0 & 0 & 0 \\
0 & t & 0 & 0 & t+1 \\
0 & 0 & 0 & 5 & 0 \\
0 & 0 & 0 & 1 & 5
\end{pmatrix} \sim
\begin{cases}
    \begin{pmatrix}
    0 & 0 & 1 & -1 & 0 \\
    0 & 0 & 0 & 5 & 0 \\
    0 & 0 & 0 & 0 & 1 \\
    0 & 0 & 0 & 0 & 0 \\
    0 & 0 & 0 & 0 & 0
    \end{pmatrix},     \;  t = 0 \Rightarrow \dim \ker (A_t + 2E) = 2 \\[40pt]
    \begin{pmatrix}
    0 & t & 0 & 0 & t+1 \\
    0 & 0 & 1 & t-1 & 0 \\
    0 & 0 & 0 & 5 & 0 \\
    0 & 0 & 0 & 0 & 5 \\
    0 & 0 & 0 & 0 & 0
    \end{pmatrix},     \;  t \neq 0 \Rightarrow \dim \ker(A_t + 2E) = 1
\end{cases}
$$

Тогда при $t = 0$ получаем, что клетки будут размера 1-2, и при $t \neq 0$ получаем, что будет одна клетка размера 3.

\noindent\rule{\textwidth}{0.5pt}
Как выглядят все матрицы $D$, что коммутируют с ЖНФ? (Это было на одном из семов), это будут блочно-диагональные матрицы такие(блоки размера такого же что и в ЖНФ), что каждый блок коммутирует с жордановой клеткой этого блока. А коммутирование с жордановой клеткой тоже было в ДЗ, это будет матрица вида:

$$
\begin{pmatrix}
a_0   & a_1   & a_2   & \cdots & a_{n-1} \\
0     & a_0   & a_1   & \cdots & a_{n-2} \\
\vdots&\ddots&\ddots&\ddots & \vdots  \\
0     & \cdots& 0     & a_0    & a_1     \\
0     & \cdots& \cdots& 0      & a_0
\end{pmatrix}
$$

Чтобы $D$(конкретный блок) при этом был диагонализуемым, (так как $\chi_D(\lambda) = (\lambda - a_0)^n$), должно быть, что $\dim \ker(D - a_0E) = n \Rightarrow$ $D$ - скалярная матрица.

Тогда при $t = 0$:

$$
D =
\begin{pmatrix}
\begin{smallmatrix}a&0\\0&a\end{smallmatrix} 
&0&0\\
0&c&0\\
0&0&
\begin{smallmatrix}d&0\\0&d\end{smallmatrix}
\end{pmatrix} \Rightarrow \dim K_0 = 3 = \dim L_0
$$

И при $t \neq 0$:

$$
D =
\begin{pmatrix}
\begin{matrix}a&0&0\\0&a&0\\0&0&a\end{matrix}
& 0\\
0&
\begin{smallmatrix}d&0\\0&d\end{smallmatrix}
\end{pmatrix} \Rightarrow \dim K_t = 2 = \dim L_t
$$

То есть наибольшая размерность $L_t$ достигается при $t = 0$.

\subsection*{№ 5}
Так как $\beta$ - симметричная биленейная форма, то по утв. 164 получим, что существует такой базис, что $\beta$ в этом базисе имеет диагональный вид.

Но нас спрашивают можно ли дополнить $v$ до базиса так, чтобы $\beta$ в нем была диагональной, то есть выбрать такие $e_1, e_2, e_3, e_4$, что $\beta(e_i, e_j) = 0$ при $i \neq j$ и один из них равен $v$.

Поскольку

$$
\beta(v,v)=v^T B\,v
=\begin{pmatrix}2&4&-1&-2\end{pmatrix}
\begin{pmatrix}
0 & 0 & 5 & -3\\
0 & 0 & -3 & 2\\
5 & -3 & -7 & 1\\
-3 & 2 & 1 & 2
\end{pmatrix}
\begin{pmatrix}2\\4\\-1\\-2\end{pmatrix}
=1\neq0,
$$

То $e_1$ можно выбрать равным $v$. А значит ограничение $\beta$ на $\langle v \rangle$ ненулевое, а значит $R^4 = \langle v \rangle + \langle v \rangle^\perp$, а дальше можно воспользоваться утверждением 164, то есть ответ да, можно.

Давайте найдем ортогональное дополнение к $\langle v \rangle$, то есть $x : v^T B x = 0$, тогда базисом $\langle v \rangle^\perp$ будет ФСР этой системы.

$$
v^TB =
\begin{pmatrix}2&4&-1&-2\end{pmatrix}
\begin{pmatrix}
0 & 0 & 5 & -3\\
0 & 0 & -3 & 2\\
5 & -3 & -7 & 1\\
-3 & 2 & 1 & 2
\end{pmatrix} =
\begin{pmatrix}
1 & -1 & 3 & -3
\end{pmatrix}
$$

Отсюда ФСР:

$$
\begin{gathered}
v_2 = (*, 0, 0, 1)^T = (3, 0, 0, 1)^T\\[6pt]
v_3 = (*, 0, 1, 0)^T = (-3, 0, 1, 0)^T\\[6pt]
v_4 = (*, 1, 0, 0)^T = (1, 1, 0, 0)^T
\end{gathered}
$$

Теперь применив ортогонализацию Грама-Шмидта, получим такой набор $u_2, u_3, u_4 : \langle v_2, v_3, v_4 \rangle = \langle u_2, u_3, u_4 \rangle$, что новые вектора попарно ортогональны.

$$
\begin{gathered}
u_2 = v_2 = (3, 0, 0, 1)^T\\[6pt]
u_3' = v_3 - \frac{\beta(v_3, u_2)}{\beta(u_2, u_2)}u_2 = (-3, 0, 1, 0)^T + \frac{25}{16}(3, 0, 0, 1)^T = (\frac{27}{16}, 0, 1, \frac{25}{16})^T
\end{gathered}
$$

Давайте домножим на 16, тем самым избавимся от дробей, ортогональность с $u_2$ при этом не потеряется.

$$
\begin{gathered}
u_3 = 16u_3' = (27, 0, 16, 25)^T\\[6pt]
u'_4 = v_4 - \frac{\beta(v_4, u_2)}{\beta(u_2, u_2)}u_2 - \frac{\beta(v_4, u_3)}{\beta(u_3, u_3)}u_3 = (1, 1, 0, 0)^T - \frac{1}{16}(3, 0, 0, 1)^T - \frac{7}{528}(27, 0, 16, 25)^T = (\frac{5}{11}, 1, -\frac{7}{33}, -\frac{13}{33})^T\\[6pt]
u_4 = 33u'_4 = (15, 33, -7, -13)^T
\end{gathered}
$$

Тогда выбрав $e_1 = v, e_2 = u_2, e_3 = u_3, e_4 = u_4$ получим требуемое.

Форма $\beta$ в этом базисе будет иметь вид:

$$
\begin{pmatrix}
\beta(e_1, e_1) & 0 & 0 & 0\\
0 & \beta(e_2, e_2) & 0 & 0\\
0 & 0 & \beta(e_3, e_3) & 0\\
0 & 0 & 0 & \beta(e_4, e_4)
\end{pmatrix} = 
\begin{pmatrix}
1 & 0 & 0 & 0\\
0 & -16 & 0 & 0\\
0 & 0 & 528 & 0\\
0 & 0 & 0 & -33
\end{pmatrix}
$$


\end{document}